% TeX шаблон, пример оформления отчёта по лабораторной работе.
% Автор: Шмаков И.А.
% Дата: 02 февраля 2018 года.
% Сборка документа из командной строки:
% ~$ pdflatex -shell-escape main.tex

\documentclass[a4paper,14pt]{extarticle}
\usepackage[utf8]{inputenc}
\usepackage[english,russian]{babel}
% Пакет отвечающий за рисование графиков в нутри TeX файла.
% https://ru.sharelatex.com/blog/2013/08/27/tikz-series-pt1.html
% https://ru.sharelatex.com/learn/TikZ_package
% http://tex.stackexchange.com/questions/300945/how-to-draw-this-picture-with-latex-tikz-pgf
\usepackage{tikz}

\usepackage{setspace}
\singlespacing % одинарный интервал

\usepackage{amsmath}
\usepackage{amsfonts}
\usepackage{amssymb}
\usepackage{mathtext}
\usepackage{graphicx}
\usepackage{float}
\usepackage[left=3cm, right=1cm, top=1.5cm, bottom=1.5cm]{geometry}
\usepackage{icomma} % "Умная" запятая: $0,2$ --- число, $0, 2$ --- перечисление
\usepackage{indentfirst} % Красная строка.
% Пакет отвечающий за листинги.
\usepackage{minted}
% Русский язык в листингах minted
% http://aakinshin.blogspot.ru/2014/01/latex-minted.html
% \makeatletter
% \newcommand{\minted@write@detok}[1]{%
%   \immediate\write\FV@OutFile{\detokenize{#1}}}%
% 
% \newcommand{\minted@FVB@VerbatimOut}[1]{%
%   \@bsphack
%   \begingroup
%     \FV@UseKeyValues
%     \FV@DefineWhiteSpace
%     \def\FV@Space{\space}%
%     \FV@DefineTabOut
%     %\def\FV@ProcessLine{\immediate\write\FV@OutFile}% %Old, non-Unicode version
%     \let\FV@ProcessLine\minted@write@detok %Patch for Unicode
%     \immediate\openout\FV@OutFile #1\relax
%     \let\FV@FontScanPrep\relax
% %% DG/SR modification begin - May. 18, 1998 (to avoid problems with ligatures)
%     \let\@noligs\relax
% %% DG/SR modification end
%     \FV@Scan}
%     \let\FVB@VerbatimOut\minted@FVB@VerbatimOut
% 
% \renewcommand\minted@savecode[1]{
%   \immediate\openout\minted@code\jobname.pyg
%   \immediate\write\minted@code{\expandafter\detokenize\expandafter{#1}}%
%   \immediate\closeout\minted@code}
% \makeatother
% Русский язык в листингах minted

\renewcommand{\thesection}{\arabic{section}}

% Описание титульной страницы
%\title{ГОУ ВО <<Алтайский государственный университет>>}
%\author{Шмаков И.А.}


\begin{document}
\begin{titlepage}
  \begin{center}
    ГОУ ВО АЛТАЙСКИЙ ГОСУДАРСТВЕННЫЙ УНИВЕРСИТЕТ
    \vspace{0.25cm}
    
    Физико-технический факультет
    
    Кафедра вычислительной техники и электроники
    \vfill
    
    {\LARGE Базовые команды для работы к командной строке}\\[5mm]
    \textsc{(Отчёт по индивидуальному заданию по курсу <<Операционные системы>>)}
  \bigskip

\end{center}
\vfill

\newlength{\ML}
\settowidth{\ML}{«\underline{\hspace{0.7cm}}» \underline{\hspace{2cm}}}
\hfill\begin{minipage}{0.4\textwidth}
  Выполнил студент 2-го курса, 585 группы:\\
  \underline{\hspace{\ML}} А.\,И.~Селянин\\
  «\underline{\hspace{0.7cm}}» \underline{\hspace{2cm}} \the\year~г.
\end{minipage}%
\bigskip

\hfill\begin{minipage}{0.4\textwidth}
  Проверил\\
  \underline{\hspace{\ML}} П.\,Н.~Уланов\\
  «\underline{\hspace{0.7cm}}» \underline{\hspace{2cm}} \the\year~г.
\end{minipage}%
\vfill

\begin{center}
  Барнаул, \the\year~г.
\end{center}
\end{titlepage}

%\maketitle

\tableofcontents

\section{Введение и постановка задачи}
Ознакомиться с командами представленными выше. Привести краткое описание к каждой команде.

\section{Теоретическое описание задачи}
В данном пункте отчёта требуется описать теоретическую часть работы.

\section{Алгоритм и блок-схема}
В данном пункте отчёта требуется предоставить алгоритм работы программы и блок-схему, 
а также информацию о правилах в соответствии с которыми были построены алгоритм и 
блок-схема.

\section{Проверка работы программы}
В данном пункте отчёта требуется предоставить данные по работе программы. В частности
сравнение ваших результатов для какой-либо приближенной функции и <<оригинальной>> функции.
\begin{figure}[H]
\centering 
\begin{tikzpicture}
\draw (0,0) parabola (5,5);
\draw (0,0) .. controls (0,4) and (4,0) .. (4,5);
\draw[step=1cm,gray,very thin] (-1.5,-1.5) grid (5.9,5.9);
\draw[thick,->] (0,0) -- (5.5,0) node[anchor=north west] {x};
\draw[thick,->] (0,0) -- (0,5.5) node[anchor=south east] {y};

\foreach \x in {0,1,2,3,4,5}
    \draw (\x cm,1pt) -- (\x cm,-1pt) node[anchor=north] {$\x$};
\foreach \y in {0,1,2,3,4,5}
    \draw (1pt,\y cm) -- (-1pt,\y cm) node[anchor=east] {$\y$};
    
\end{tikzpicture}
\caption{Пример использования Tikz} 
\end{figure}

\section{Вывод по работе}
В данном пункте отчёта требуется сделать вывод по проделанной работе, 
сделать оценку полученной погрешности.

\renewcommand\bibname{Список литературы}
\addcontentsline{toc}{section}{Список литературы}
\begin{thebibliography}{99}
\bibitem{catheydow} W.T. Cathey and E.R. Dowski, <<New paradigm for imaging systems>>, Appl. Opt. 41, pp. 6080-6092, 2002.
\end{thebibliography}

\addcontentsline{toc}{section}{Приложение}
\section*{Приложение}
\begin{minted}[mathescape,linenos,frame=lines]{python}
import numpy as np
 
def incmatrix(genl1,genl2):
    m = len(genl1)
    n = len(genl2)
    M = None #to become the incidence matrix
    VT = np.zeros((n*m,1), int)  #dummy variable
 
    #compute the bitwise xor matrix
    M1 = bitxormatrix(genl1)
    M2 = np.triu(bitxormatrix(genl2),1) 
 
    for i in range(m-1):
        for j in range(i+1, m):
            [r,c] = np.where(M2 == M1[i,j])
            for k in range(len(r)):
                VT[(i)*n + r[k]] = 1;
                VT[(i)*n + c[k]] = 1;
                VT[(j)*n + r[k]] = 1;
                VT[(j)*n + c[k]] = 1;
 
                if M is None:
                    M = np.copy(VT)
                else:
                    M = np.concatenate((M, VT), 1)
 
                VT = np.zeros((n*m,1), int)
 
    return M
\end{minted}

\begin{minted}[mathescape,linenos,frame=lines]{C}
#include "stdio.h"

int main(){
  int a, b, c; 
  
  a = 5;
  b = 10;
  c = a + b;
  
  printf("Результат сложения a + b = %i\n", c);
  
  return 0;
}
\end{minted}

\end{document}          
